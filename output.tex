
\documentclass{ctexbook}  % 使用ctexbook文档类,支持中文并允许使用\chapter
\usepackage[utf8]{inputenc}
\usepackage{fancyhdr}
\usepackage{fancyvrb}
\usepackage{tocloft}
\usepackage{geometry}
\geometry{a4paper, margin=1in}

% 页眉和页脚
\pagestyle{fancy}
\fancyhf{}
\rhead{\rightmark}  % 页眉显示当前小标题
\lhead{XCPC 算法模板}
\cfoot{\thepage}

% 封面设置
\title{XCPC 算法模板}
\author{MaxDYF}
\date{\today}

\begin{document}

\maketitle
\newpage

% 目录生成
\tableofcontents
\newpage
\chapter{.vscode}
\subsection{CF模板}
\lstset{basicstyle=	tfamily}
\begin{Verbatim}[fontsize=\small]
// #pragma GCC optimize("Ofast,no-stack-protector")
#include <bits/stdc++.h>
typedef long long ll;
typedef double db;
typedef pair<int, int> pii;
typedef pair<ll, ll> pll;
typedef pair<db, db> pdd;
typedef pair<ll, int> pli;
using namespace std;

const int N = 2e5 + 10;
const int inf = 1 << 30;
const ll inf64 = 1ll << 60;
const double PI = acos(-1);

#define lowbit(x) (x & -x)

int n, m, k, q;

void work()
{
    cin >> n;
}
int main()
{
    cin.tie(nullptr)->sync_with_stdio(false);
    int t;
    cin >> t;
    while (t-- > 0)
    {
        work();
    }
}
\end{Verbatim}

\chapter{图论}
\section{最短路}
\subsection{Dijkstra}
\lstset{basicstyle=	tfamily}
\begin{Verbatim}[fontsize=\small]
#include <bits/stdc++.h>
const int N = 2e5 + 10;

namespace Graph
{
    typedef int64_t T;
    typedef std::pair<int64_t, int32_t> pli;
    std::vector<pli> to[N];
    void add(int x, int y, T val)
    {
        to[x].push_back({val, y});
    }
    bool vis[N];
    T dis[N];
    void shortest_path(int32_t from)
    {
        // using Dijkstra
        std::priority_queue<pli, std::vector<pli>, std::greater<pli>> q;
        q.push(std::make_pair(0, from));
        memset(dis, 0x3f, sizeof dis);
        memset(vis, 0, sizeof vis);
        dis[from] = 0;
        while (!q.empty())
        {
            auto [nowval, x] = q.top();
            q.pop();
            if (vis[x])
                continue;
            vis[x] = 1;
            for (auto [val, y] : to[x])
                if (dis[y] > dis[x] + val)
                {
                    dis[y] = dis[x] + val;
                    q.push(std::make_pair(dis[y], y));
                }
        }
    }
}

\end{Verbatim}

\section{网络流}
\subsection{Dinic.最大流}
\lstset{basicstyle=	tfamily}
\begin{Verbatim}[fontsize=\small]
#include <bits/stdc++.h>
namespace Dinic
{

    using T = int32_t;
    // 定义储存流量的类型,默认为int32_t
    const int CNT_NODE = 2e5 + 10;
    // 定义节点Node的数量
    const int CNT_EDGE = 2e5 + 10;
    // 定义边的数量
    struct Edge
    {
        int to;
        int nxt;
        T flow;
    } e[CNT_EDGE];
    int head[CNT_NODE], cnt = 1;
    int n;
    void add(int x, int y, T flow)
    {
        e[++cnt] = Edge{y, head[x], flow};
        head[x] = cnt;
        e[++cnt] = Edge{x, head[y], 0};
        head[y] = cnt;
        n = std::max(n, x);
        n = std::max(n, y);
    }
    int st, ed;
    T dis[CNT_NODE];
    int cur[CNT_NODE];
    bool bfs()
    {
        std::queue<int> que;
        que.push(ed);
        memset(dis, 0, sizeof(dis));
        dis[ed] = 1;
        while (!que.empty())
        {
            auto x = que.front();
            que.pop();
            cur[x] = head[x];
            for (auto i = head[x]; i; i = e[i].nxt)
            {
                auto y = e[i].to;
                if (!dis[y] && e[i ^ 1].flow)
                {
                    dis[y] = dis[x] + 1;
                    que.push(y);
                }
            }
        }
        return dis[st];
    }
    T dfs(int x, T flow)
    {
        auto rest = flow;
        if (!flow || x == ed)
            return flow;
        for (auto &i = cur[x]; i && rest; i = e[i].nxt)
        {
            auto y = e[i].to;
            if (dis[y] == dis[x] - 1 && e[i].flow)
            {
                auto k = dfs(y, std::min(rest, e[i].flow));
                e[i].flow -= k;
                e[i ^ 1].flow += k;
                rest -= k;
            }
        }
        return flow - rest;
    }
    T dinic()
    {
        for (int i = 1; i <= n; i++)
            cur[i] = head[i];
        T ans = 0, k;
        while (bfs())
            while (k = dfs(st, 0x3f3f3f3f))
                ans += k;
        return ans;
    }
}
\end{Verbatim}

\subsection{Dinic费用流}
\lstset{basicstyle=	tfamily}
\begin{Verbatim}[fontsize=\small]
#include <bits/stdc++.h>
namespace Dinic
{
    using T = int64_t;
    const int NODE = 1e5 + 10;
    const int EDGE = 1e5 + 10;
    struct Edge
    {
        int to, nxt;
        T flow, cost;
        // flow剩余流量
        // cost通过费用
    } edge[EDGE];
    int head[NODE];      // 邻接表头
    int cnt = 1;         // 总边数
    int pre[NODE];       // 表示当前增广路中每个节点的上一个节点
    T flow[NODE];        // 到达当前节点的总流量
    T cost[NODE];        // 到达当前节点的总花费
    int last[NODE];      // 到达上一节点的边的编号
    std::queue<int> que; // SPFA用队列
    bool vis[NODE];      // 当前节点是否走过
    int start, end;
    void add(int x, int y, T flow, T cost)
    {
        edge[++cnt] = Edge{y, head[x], flow, cost};
        head[x] = cnt;
        edge[++cnt] = Edge{x, head[y], 0, -cost};
        head[y] = cnt;
    }
    bool spfa()
    {
        memset(flow, 0x3f, sizeof flow);
        memset(cost, 0x3f, sizeof cost);
        memset(vis, 0, sizeof vis);
        que.push(start);
        pre[end] = -1;
        cost[start] = 0;
        vis[start] = 1;
        while (!que.empty())
        {
            int x = que.front();
            que.pop();
            vis[x] = 0;
            for (int i = head[x]; i; i = edge[i].nxt)
            {
                int y = edge[i].to;
                if (edge[i].flow > 0 && cost[y] > cost[x] + edge[i].cost)
                {
                    flow[y] = std::min(flow[x], edge[i].flow);
                    cost[y] = cost[x] + edge[i].cost;
                    last[y] = i;
                    pre[y] = x;
                    if (!vis[y])
                    {
                        vis[y] = 1;
                        que.push(y);
                    }
                }
            }
        }
        return pre[end] != -1;
    }
    std::pair<T, T> work()
    {
        int maxflow = 0, mincost = 0;
        while (spfa())
        {
            maxflow += flow[end];
            mincost += cost[end] * flow[end];
            int now = end;
            while (now != start)
            {
                edge[last[now]].flow -= flow[end];
                edge[last[now] ^ 1].flow += flow[end];
                now = pre[now];
            }
        }
        return {maxflow, mincost};
    }
}

\end{Verbatim}

\chapter{字符串处理}
\subsection{KMP}
\lstset{basicstyle=	tfamily}
\begin{Verbatim}[fontsize=\small]
#include <bits/stdc++.h>
using namespace std;
const int N = 1e6;
char a[N], b[N];
int p[N];
int main()
{
    int n, m;
    cin >> (a + 1) >> (b + 1);
    n = strlen(a + 1);
    m = strlen(b + 1);
    for (int i = 2, j = 0; i <= m; i++)
    {
        while (j && b[i] != b[j + 1])
            j = p[j];
        if (b[j + 1] == b[i])
            j++;
        p[i] = j;
    }
    for (int i = 1, j = 0; i <= n; i++)
    {
        while (j && a[i] != b[j + 1])
            j = p[j];
        if (a[i] == b[j + 1])
            j++;
        if (j == m)
        {
            cout << i - m + 1 << '\textbackslash{}n';
            j = p[j];
        }
    }
    for (int i = 1; i <= m; i++)
        cout << p[i] << ' ';
}
\end{Verbatim}

\subsection{Manacher}
\lstset{basicstyle=	tfamily}
\begin{Verbatim}[fontsize=\small]
int manacher(string s)
{
    string str;
    str += "*#";
    for (auto ch : s)
    {
        str.push_back(ch);
        str.push_back('#');
    }
    str += "#)";
    int n = str.length() - 1;
    vector<int> p((int)str.length());
    int maxid = 1, id = 1, ans = 0;
    for (int i = 1; i < n; i++)
    {
        if (i < maxid)
            p[i] = min(maxid - i, p[2 * id - i]);
        else
            p[i] = 1;
        while (str[i - p[i]] == str[i + p[i]])
            p[i]++;
        if (maxid < i + p[i])
        {
            maxid = i + p[i];
            id = i;
        }
        ans = max(ans, p[i] - 1);
    }
    return ans;
}
\end{Verbatim}

\subsection{Trie}
\lstset{basicstyle=	tfamily}
\begin{Verbatim}[fontsize=\small]
#include <algorithm>
#include <cstdio>
#include <map>
#include <vector>
#include <string>
typedef long long i64;
class Trie
{
public:
    Trie() : m_size(1)
    {
        tree.clear();
        tree.push_back(Node(-1, 0));
    }
    ~Trie() {}
    int size() const { return m_size; }
    int insert(char *s, int len)
    {
        int now = 0;
        for (int i = 0; i < len; ++i)
        {
            if (!tree[now].nxt.count(s[i]))
            {
                tree[now].nxt[s[i]] = m_size++;
                tree.push_back(Node(now, s[i]));
            }
            now = tree[now].nxt[s[i]];
        }
        return now;
    }
    friend class SAM;

private:
    struct Node
    {
        int fa;
        char c;
        std::map<char, int> nxt;
        Node(int _1, char _2) : fa(_1), c(_2)
        {
            nxt.clear();
        }
    };
    std::vector<Node> tree;
    int m_size;
};
\end{Verbatim}

\subsection{exKMP}
\lstset{basicstyle=	tfamily}
\begin{Verbatim}[fontsize=\small]
// #pragma GCC optimize(2)
#include <bits/stdc++.h>
using namespace std;

const int N = 2e5 + 10;
const int inf = 1 << 30;
const long long llinf = 1ll << 60;
const double PI = acos(-1);

#define lowbit(x) (x & -x)
typedef long long ll;
typedef double db;
typedef pair<int, int> pii;
typedef pair<ll, ll> pll;
typedef pair<db, db> pdd;
typedef pair<ll, int> pli;

long long n, m, k, q, x;

/*
 * Z函数模板
 * 实现了基础的Z函数功能
 * 注意,根据定义,z[0]=0,而非z[0]=n
 * 通过例题: https://www.luogu.com.cn/problem/P10320?contestId=156706
 */

namespace Zfunc
{
    vector<int> getZ(string s)
    {
        int n = s.length();
        vector<int> z(n);
        int l = 0, r = 0;
        for (int i = 0; i < n; i++)
        {
            // 如果不超出范围,则直接继承
            if (i <= r && i + z[i - l] - 1 < r)
                z[i] = z[i - l];
            // 否则,就尝试从r-l+1长度开始暴力扩展
            else
            {
                z[i] = max(0, r - i + 1);
                while (i + z[i] < n && s[i + z[i]] == s[z[i]])
                    ++z[i];
            }
            // 更新l, r的边界
            if (i + z[i] - 1 > r)
                l = i, r = i + z[i] - 1;
        }
        return z;
    }
}
void work()
{
    string a, b;
    cin >> a >> b;
    int n = a.length(), m = b.length();
    string c = b + '*' + a;
    auto z1 = Zfunc::getZ(b),
         z2 = Zfunc::getZ(c);
    z1[0] = m;
    long long ans1 = 0, ans2 = 0;
    for (int i = 0; i < m; i++)
        ans1 ^= (ll)(i + 1) * (ll)(z1[i] + 1);
    for (int i = m + 1; i < n + m + 1; i++)
        ans2 ^= (ll)(i - m) * (ll)(z2[i] + 1);
    cout << ans1 << endl
         << ans2 << endl;
}
int main()
{
    ios::sync_with_stdio(0);
    cin.tie(0);
    cout.tie(0);
    work();
}
\end{Verbatim}

\subsection{字符串哈希}
\lstset{basicstyle=	tfamily}
\begin{Verbatim}[fontsize=\small]
// #pragma GCC optimize(2)
#include <bits/stdc++.h>
typedef std::pair<int, int> HashPair;
class Hash
{
private:
    const int CHAR_SIZE = 26; // 字符集大小
    const int CC = (CHAR_SIZE + 1LL);
    std::vector<HashPair> hashPow;
    std::vector<HashPair> hashVal;
    const int mod1 = 100000007,
              mod2 = 100000037;
    /*
     *    字符转换函数
     *    将字符转换为Hash中的编号
     */
    int transChar(char ch)
    {
        return (ch - 'A' + 1);
    }
    /*
     *    序列生成Hash生成函数并储存
     */
    void makeHash(std::string &s)
    {
        size_t len = s.length();
        hashPow.clear();
        hashVal.clear();
        hashPow.resize(len);
        hashVal.resize(len);
        hashPow[0] = {1LL, 1LL};
        for (int i = 1; i < len; i++)
        {
            hashPow[i].first = (int64_t)hashPow[i - 1].first * CC % mod1;
            hashPow[i].second = (int64_t)hashPow[i - 1].second * CC % mod2;
        }
        for (size_t i = 0; i < len; i++)
        {
            int p = transChar(s[i]);
            if (i > 0)
            {
                hashVal[i].first = ((int64_t)hashVal[i - 1].first * CC + p) % mod1;
                hashVal[i].second = ((int64_t)hashVal[i - 1].second * CC + p) % mod2;
            }
            else
            {
                hashVal[i].first = transChar(s[i]) % mod1;
                hashVal[i].second = transChar(s[i]) % mod2;
            }
        }
    }

public:
    Hash(std::string &str)
    {
        makeHash(str);
    }
    /*
     *  截取某一段的Hash值
     *    传入参数为[l, r]
     *  默认返回一个HashPair
     */
    HashPair subHash(int l, int r)
    {
        if (l > r)
            return {0, 0};
        if (l == 0)
            return hashVal[r];
        else
        {
            HashPair result;
            auto c1 = (int64_t)hashVal[l - 1].first * hashPow[r - l + 1].first % mod1;
            result.first = ((int64_t)hashVal[r].first - c1 + mod1) % mod1;
            auto c2 = (int64_t)hashVal[l - 1].second * hashPow[r - l + 1].second % mod2;
            result.second = ((int64_t)hashVal[r].second - c2 + mod2) % mod2;
            return result;
        }
    }
};
\end{Verbatim}

\section{自动机}
\subsection{AC自动机}
\lstset{basicstyle=	tfamily}
\begin{Verbatim}[fontsize=\small]
// #pragma GCC optimize(2)
#include <bits/stdc++.h>
using namespace std;

const int N = 2e5 + 10;
const int inf = 1 << 30;
const long long llinf = 1ll << 60;
const double PI = acos(-1);

#define lowbit(x) (x & -x)
typedef long long ll;
typedef double db;
typedef pair<int, int> pii;
typedef pair<ll, ll> pll;
typedef pair<db, db> pdd;
typedef pair<ll, int> pli;

/*
 * AC自动机模板
 * 本模板使用了建图优化以及拓扑排序
 * 最终第i个模式串匹配文本串次数为 vis[rev[i]]
 * 模板题https://www.luogu.com.cn/problem/P5357
 */
namespace ACautomaton
{
    struct TrieNode
    {
        int son[26];
        int fail;
        int flag;
        int ans;
        TrieNode()
        {
            memset(son, 0, sizeof son);
            fail = flag = ans = 0;
        }
        void clear(void)
        {
            (*this) = TrieNode();
        }
    };
    TrieNode trie[N];
    int vis[N];   // 记录能匹配到当前点的答案。
    int rev[N];   // 索引,将第i个模式索引至答案的下标,作用是模式串去重
    int indeg[N]; // getfail()所求的入度,用于拓扑排序
                  // 注意,indeg[]的入度是fail边的入度
    int cnt = 1;
    /*
     * 插入一个模式串str, 其编号为num
     */
    void insert(string &str, int num)
    {
        int u = 1;
        for (auto ch : str)
        {
            int v = ch - 'a';
            if (trie[u].son[v] == 0)
                trie[u].son[v] = ++cnt;
            u = trie[u].son[v];
        }
        if (!trie[u].flag)
            trie[u].flag = num;
        rev[num] = trie[u].flag;
    }
    /*
     * 构建fail指针
     * 为拓扑排序建立入度
     */
    void getfail(void)
    {
        std::queue<int> q;
        for (int i = 0; i < 26; i++)
            trie[0].son[i] = 1;
        trie[1].fail = 0;
        q.push(1);
        while (!q.empty())
        {
            int u = q.front();
            q.pop();
            int fail = trie[u].fail;
            for (int i = 0; i < 26; i++)
            {
                // 减去了跳fail边的操作
                // 直接将儿子节点连到fail节点的对应儿子节点
                // 相当于将fail边路径压缩
                if (trie[u].son[i] == 0)
                {
                    trie[u].son[i] = trie[fail].son[i];
                    continue;
                }
                int v = trie[u].son[i],
                    nextFail = trie[fail].son[i];
                indeg[nextFail]++;
                trie[v].fail = nextFail;
                q.push(v);
            }
        }
    }
    /*
     * 导入待查寻的文本串
     * 将匹配的路径记录到TrieNode::ans上
     * 以便拓扑排序时求解
     */
    void query(string &str)
    {
        int u = 1;
        for (auto ch : str)
        {
            u = trie[u].son[ch - 'a'];
            trie[u].ans++;
        }
    }
    /*
     * 关键步骤拓扑排序
     * 求解匹配答案
     */
    void topu(void)
    {
        std::queue<int> q;
        for (int i = 1; i <= cnt; i++)
            if (indeg[i] == 0)
                q.push(i);
        while (!q.empty())
        {
            int u = q.front();
            q.pop();
            if (trie[u].flag != 0)
                vis[trie[u].flag] = trie[u].ans;
            int fail = trie[u].fail;
            trie[fail].ans += trie[u].ans;
            --indeg[fail];
            if (indeg[fail] == 0)
                q.push(fail);
        }
    }
    /*
     *  返回第num个模式串对应的答案
     */
    int getAnswer(int num)
    {
        return vis[rev[num]];
    }
}

int n, m, k, q;

void work()
{
    cin >> n;
    string s, t;
    for (int i = 1; i <= n; i++)
    {
        cin >> s;
        ACautomaton::insert(s, i);
    }
    cin >> t;
    ACautomaton::getfail();
    ACautomaton::query(t);
    ACautomaton::topu();
    for (int i = 1; i <= n; i++)
        cout << ACautomaton::getAnswer(i) << '\textbackslash{}n';
}
int main()
{
    ios::sync_with_stdio(0);
    cin.tie(0);
    cout.tie(0);
    work();
}
\end{Verbatim}

\subsection{PAM}
\lstset{basicstyle=	tfamily}
\begin{Verbatim}[fontsize=\small]
class PAM
{
private:
    struct Node
    {
        int fail;
        int cnt;
        int len;
        int son[26];
        Node()
        {
            fail = 0;
            len = 0;
            cnt = 0;
            memset(son, 0, sizeof son);
        }
        Node(int len)
        {
            *this = Node();
            this->len = len;
        }
    };
    vector<Node> tr;
    string str;
    vector<int> id2Node;
    int lst;
    vector<vector<int>> fa;
    int getfail(int x, int i)
    {
        while (i - tr[x].len - 1 < 0 || str[i - tr[x].len - 1] != str[i])
            x = tr[x].fail;
        return x;
    }
    int newnode(int len)
    {
        tr.push_back(Node(len));
        return tr.size() - 1;
    }

public:
    PAM()
    {
        lst = 0;
        tr.clear();
        newnode(0);
        newnode(-1);
        str.push_back('#');
        tr[1].fail = 1;
        tr[0].fail = 1;
    }
    void insert(char c)
    {
        str.push_back(c);
        int i = str.length() - 1;
        int x = getfail(lst, i);
        if (!tr[x].son[c - 'a'])
        {
            int p = newnode(tr[x].len + 2);
            int f = getfail(tr[x].fail, i);
            tr[p].fail = tr[f].son[c - 'a'];
            tr[x].son[c - 'a'] = p;
            tr[p].cnt = tr[tr[p].fail].cnt + 1;
        }
        lst = tr[x].son[c - 'a'];
        id2Node.push_back(lst);
    }
    void buildFailTree()
    {
        int n = tr.size();
        fa = vector<vector<int>>(n, vector<int>(25));
        for (int x = 0; x < n; x++)
        {
            fa[x][0] = tr[x].fail;
            for (int i = 1; i < 25; i++)
                fa[x][i] = fa[fa[x][i - 1]][i - 1];
        }
    }
};
\end{Verbatim}

\subsection{SAM}
\lstset{basicstyle=	tfamily}
\begin{Verbatim}[fontsize=\small]
#include <algorithm>
#include <cstdio>
#include <map>
#include <queue>
#include <string>
#include <vector>
typedef long long i64;

class Trie
{
public:
    Trie() : m_size(1)
    {
        tree.clear();
        tree.push_back(Node(-1, 0));
    }
    virtual ~Trie() {}
    virtual int size() const { return m_size; }
    int insert(char *s, int len)
    {
        int now = 0;
        for (int i = 0; i < len; ++i)
        {
            if (!tree[now].nxt.count(s[i]))
            {
                tree[now].nxt[s[i]] = m_size++;
                tree.push_back(Node(now, s[i]));
            }
            now = tree[now].nxt[s[i]];
        }
        return now;
    }

protected:
    struct Node
    {
        int fa, lst;
        char c;
        std::map<char, int> nxt;
        Node(int _1, char _2) : fa(_1), lst(0), c(_2)
        {
            nxt.clear();
        }
    };
    std::vector<Node> tree;
    int m_size;
};

class SAM : public Trie
{
public:
    SAM() : m_size(1), _lst(0)
    {
        seq.clear();
        seq.push_back(state(0, -1));
    }
    virtual ~SAM() {}
    int size() const { return m_size; }
    void build()
    {
        std::queue<int> q;
        for (int t = 0; t < 26; ++t)
            if (~Trie::tree[0].nxt[t])
                q.push(Trie::tree[0].nxt[t]);
        Trie::tree[0].lst = 0;
        while (!q.empty())
        {
            int x = q.front();
            q.pop();
            Trie::tree[x].lst = saminsert(Trie::tree[x].c, Trie::tree[Trie::tree[x].fa].lst);
            for (int t = 0; t < 26; ++t)
                if (~Trie::tree[x].nxt[t])
                    q.push(Trie::tree[x].nxt[t]);
        }
    }
    friend i64 calc(SAM &sam);

private:
    struct state
    {
        int len, link, cnt;
        std::map<char, int> nxt;
        state(int _1, int _2, int _3 = 0) : len(_1), link(_2), cnt(_3)
        {
            nxt.clear();
        }
    };
    std::vector<state> seq;
    int m_size, _lst;
    int saminsert(char c, int lst)
    {
        if (seq[lst].nxt.count(c))
        {
            int p = lst, q = seq[lst].nxt[c];
            if (seq[q].len == seq[p].len + 1)
                return q;
            else
            {
                int r = m_size++;
                seq.push_back(state(seq[p].len + 1, seq[q].link, 0));
                seq[r].nxt = seq[q].nxt;
                seq[q].link = r;
                while (~p && seq[p].nxt.count(c) && seq[p].nxt[c] == q)
                {
                    seq[p].nxt[c] = r;
                    p = seq[p].link;
                }
                return r;
            }
        }
        int cur = m_size++;
        seq.push_back(state(seq[lst].len + 1, -1, 1));
        int p = lst;
        lst = cur;
        while (~p && !seq[p].nxt.count(c))
        {
            seq[p].nxt[c] = cur;
            p = seq[p].link;
        }
        if (p == -1)
        {
            seq[cur].link = 0;
        }
        else
        {
            int q = seq[p].nxt[c];
            if (seq[q].len == seq[p].len + 1)
            {
                seq[cur].link = q;
            }
            else
            {
                int r = m_size++;
                seq.push_back(state(seq[p].len + 1, seq[q].link, 0));
                seq[r].nxt = seq[q].nxt;
                seq[cur].link = seq[q].link = r;
                while (~p && seq[p].nxt.count(c) && seq[p].nxt[c] == q)
                {
                    seq[p].nxt[c] = r;
                    p = seq[p].link;
                }
            }
        }
        return cur;
    }
};
const int N = 1000010;
char str[N];
int buc[N << 1], pos[N << 1];
SAM sam;
i64 calc(SAM &sam)
{
    i64 ans = 0;
    for (int i = 1; i < sam.size(); ++i)
        buc[sam.seq[i].len]++;
    for (int i = 1; i < sam.size(); ++i)
        buc[i] += buc[i - 1];
    for (int i = sam.size() - 1; i; --i)
        pos[buc[sam.seq[i].len]--] = i;
    for (int i = sam.size() - 1; i; --i)
    {
        int p = pos[i];
        sam.seq[sam.seq[p].link].cnt += sam.seq[p].cnt;
        if (sam.seq[p].cnt > 1)
        {
            //	printf("%d %d\textbackslash{}n", sam.seq[p].cnt, sam.seq[p].len);
            ans = std::max(ans, 1ll * sam.seq[p].cnt * sam.seq[p].len);
        }
    }
    return ans;
}
// For the array version of nxt[], see P6139.cpp
\end{Verbatim}

\chapter{数学}
\subsection{FFT}
\lstset{basicstyle=	tfamily}
\begin{Verbatim}[fontsize=\small]
#include <algorithm>
#include <iostream>
#include <vector>
#include <cmath>

const double PI = std::acos(-1);

class Complex
{
public:
    Complex(double _real = 0, double _virtual = 0);
    double getReal() const;
    double getVirtual() const;
    Complex operator+(const Complex &b) const;
    Complex operator-(const Complex &b) const;
    Complex operator*(const Complex &b) const;

private:
    double x, y;
};
Complex::Complex(double _real, double _virtual) : x(_real), y(_virtual) {}
double Complex::getReal() const { return this->x; }
double Complex::getVirtual() const { return this->y; }
Complex Complex::operator+(const Complex &b) const
{
    return Complex(this->x + b.x, this->y + b.y);
}
Complex Complex::operator-(const Complex &b) const
{
    return Complex(this->x - b.x, this->y - b.y);
}
Complex Complex::operator*(const Complex &b) const
{
    return Complex(this->x * b.x - this->y * b.y, this->x * b.y + this->y * b.x);
}

/* recursive FFT
void FFT(std::vector<Complex>& A, int flag = 1){
    int n = A.size();
    if(n == 1) return;
    std::vector<Complex> A1(n >> 1), A2(n >> 1);
    for(int i = 0; i < (n >> 1); ++i)
        A1[i] = A[i << 1],
        A2[i] = A[i << 1 | 1];
    FFT(A1, flag); FFT(A2, flag);
    Complex w1(std::cos(2.0 * PI / n), std::sin(2.0 * PI / n) * flag), wk(1, 0);
    for(int k = 0; k < (n >> 1); wk = wk * w1, ++k)
        A[k] = A1[k] + A2[k] * wk,
        A[k + (n >> 1)] = A1[k] - A2[k] * wk;
}
*/

// make reversed binary representation array
std::vector<int> makerev(const int &len)
{
    std::vector<int> ans;
    ans.emplace_back(0);
    ans.emplace_back(len >> 1);
    int l = 0;
    while ((1 << l) < len)
        l++;
    for (int i = 2; i < len; ++i)
        ans.emplace_back(ans[i >> 1] >> 1 | (i & 1) << (l - 1));
    /*
        ans[i >> 1] is the reversed representation of i >> 1
        (i >> 1) << 1 = i, so in reversed representation we need ans[i >> 1] >> 1
        if i & 1 == 1, then the MSB of reversed representation should be 1
        that is (i & 1) << (l - 1)
    */
    return ans;
}

// iterative FFT
void FFT(std::vector<Complex> &A, const int &flag = 1)
{
    static std::vector<int> rev;
    int n = A.size();
    if (int(rev.size()) != n)
        rev.clear(),
            rev = makerev(n);
    for (int i = 0; i < n; ++i)
        if (rev[i] > i)
            std::swap(A[i], A[rev[i]]);
    for (int len = 2, m = 1; len <= n; m = len, len <<= 1)
    {
        Complex w1(std::cos(2.0 * PI / len), std::sin(2.0 * PI / len) * flag), wk;
        for (int l = 0, r = len; r <= n; l += len, r += len)
        {
            wk = Complex(1, 0);
            for (int k = l; k < l + m; wk = wk * w1, ++k)
            {
                Complex x = A[k] + A[k + m] * wk,
                        y = A[k] - A[k + m] * wk;
                A[k] = x;
                A[k + m] = y;
            }
        }
    }
}

signed main(int argc, char **argv)
{
    std::cin.tie(nullptr)->sync_with_stdio(false);
    int n, m, len = 1;
    std::cin >> n >> m;
    while (len <= (n + m))
        len <<= 1; // to make the length a power of 2
    std::vector<Complex> A(len), B(len);
    for (int i = 0, x; i <= n; ++i)
    {
        std::cin >> x;
        A[i] = Complex(x, 0);
    }
    for (int i = 0, x; i <= m; ++i)
    {
        std::cin >> x;
        B[i] = Complex(x, 0);
    }
    FFT(A);
    FFT(B);
    for (int i = 0; i < len; ++i)
        A[i] = A[i] * B[i];
    FFT(A, -1);
    for (int i = 0; i <= (n + m); ++i)
        std::cout << int(A[i].getReal() / len + 0.5) << ' ';
    std::cout << std::endl;
    return 0;
}
\end{Verbatim}

\subsection{NTT}
\lstset{basicstyle=	tfamily}
\begin{Verbatim}[fontsize=\small]
#include <algorithm>
#include <iostream>
#include <vector>

typedef long long i64;

/// quick power
/// @brief
/// @tparam MOD
/// @param x
/// @param y
/// @return
template <i64 MOD>
i64 qpow(i64 x, i64 y)
{
    i64 ans = 1;
    while (y)
    {
        if (y & 1)
            ans = ans * x % MOD;
        x = x * x % MOD;
        y >>= 1;
    }
    return ans;
}

// make reversed binary representation array
/// @brief
/// @param rev
/// @param len
void makerev(std::vector<int> &rev, const int &len)
{
    rev.resize(len);
    rev[0] = 0;
    rev[1] = len >> 1;
    int l = 0;
    while ((1 << l) < len)
        l++;
    for (int i = 2; i < len; ++i)
        rev[i] = (rev[i >> 1] >> 1) | ((i & 1) << (l - 1));
    /*
        rev[i >> 1] is the reversed representation of i >> 1
        (i >> 1) << 1 = i, so in reversed representation we need rev[i >> 1] >> 1
        if i & 1 == 1, then the MSB of reversed representation should be 1
        that is (i & 1) << (l - 1)
    */
}

// iterative NTT
/// @brief
/// @tparam MOD
/// @param A
/// @param flag
template <i64 MOD>
void NTT(std::vector<i64> &A, const int &flag = 1)
{
    static std::vector<int> rev;
    static const i64 prt = 3, invprt = qpow<MOD>(prt, MOD - 2);
    int n = A.size();
    if (int(rev.size()) != n)
        makerev(rev, n);
    for (int i = 0; i < n; ++i)
        if (rev[i] > i)
            std::swap(A[i], A[rev[i]]);
    for (int len = 2, m = 1; len <= n; m = len, len <<= 1)
    {
        i64 w1 = qpow<MOD>(flag == 1 ? prt : invprt, (MOD - 1) / len), wk;
        for (int l = 0, r = len, k; r <= n; l += len, r += len)
        {
            for (k = l, wk = 1; k < l + m; wk = wk * w1 % MOD, ++k)
            {
                i64 x = A[k], y = A[k + m] * wk % MOD;
                A[k] = (x + y) % MOD;
                A[k + m] = (x - y + MOD) % MOD;
            }
        }
    }
}

const i64 mod = 998244353;

signed main(int argc, char **argv)
{
    std::cin.tie(nullptr)->sync_with_stdio(false);
    int n, m, len = 1;
    std::cin >> n >> m;
    while (len <= (n + m))
        len <<= 1; // to make the length a power of 2
    std::vector<i64> A(len), B(len);
    for (int i = 0; i <= n; ++i)
        std::cin >> A[i];
    for (int i = 0; i <= m; ++i)
        std::cin >> B[i];
    NTT<mod>(A);
    NTT<mod>(B);
    for (int i = 0; i < len; ++i)
        A[i] = A[i] * B[i] % mod;
    NTT<mod>(A, -1);
    i64 invlen = qpow<mod>(len, mod - 2);
    for (int i = 0; i <= (n + m); ++i)
        std::cout << A[i] * invlen % mod << ' ';
    std::cout << std::endl;
    return 0;
}
\end{Verbatim}

\subsection{矩阵}
\lstset{basicstyle=	tfamily}
\begin{Verbatim}[fontsize=\small]

#include <bits/stdc++.h>

/*
 * 矩阵类
 * 实现了矩阵加法、减法、乘法以及取模
 * 当MOD=0时相当于不取模
 */
template <class T, T MOD = 0>
class Matrix
{
private:
    int32_t R, C;
    std::vector<std::vector<T>> a;

public:
    Matrix() {}
    /*
     * 需要在初始化时传入其行数Rt与列数Ct
     * 可选参数为矩阵初始初始值init_val
     */
    Matrix(const int Rt, const int Ct, T init_val = 0)
    {
        a = std::vector<std::vector<T>>(Rt, std::vector<T>(Ct, init_val));
        R = Rt;
        C = Ct;
    }
    Matrix(std::vector<std::vector<T>> b)
    {
        a = b;
        R = a.size();
        C = a[0].size();
    }
    Matrix<T, MOD> operator*(const Matrix<T, MOD> b)
    {
        try
        {
            if (C != b.R)
                throw "SIZE DIFF";
        }
        catch (std::string code)
        {
            std::cerr << "MATRIX ERROR @ OPERATOR * : " + code << std::endl;
            exit(0);
        }
        Matrix<T, MOD> c(R, b.C, 0);
        for (int i = 0; i < R; i++)
            for (int j = 0; j < b.C; j++)
                for (int k = 0; k < C; k++)
                {
                    if (MOD != 0)
                        c.a[i][j] = (c.a[i][j] + (a[i][k] * b.a[k][j]) % MOD) % MOD;
                    else
                        c.a[i][j] = (c.a[i][j] + a[i][k] * b.a[k][j]);
                }
        return c;
    }
    Matrix<T, MOD> operator+(const Matrix<T, MOD> b)
    {
        try
        {
            if (R != b.R || C != b.C)
                throw "SIZE DIFF";
        }
        catch (std::string code)
        {
            std::cerr << "MATRIX ERROR @ OPERATOR + : " + code << std::endl;
            exit(0);
        }
        Matrix<T, MOD> c(R, C, 0);
        for (int i = 0; i < R; i++)
            for (int j = 0; j < C; j++)
                if (MOD != 0)
                    c.a[i][j] = (a[i][j] + b.a[i][j]) % MOD;
                else
                    c.a[i][j] = (a[i][j] + b.a[i][j]);

        return c;
    }
    Matrix<T, MOD> operator*(const T val)
    {
        Matrix<T, MOD> c(a);
        for (int i = 0; i < R; i++)
            for (int j = 0; j < C; j++)
                if (MOD != 0)
                    c.a[i][j] = (c.a[i][j] * val) % MOD;
                else
                    c.a[i][j] = (c.a[i][j] * val);
        return c;
    }
    Matrix<T, MOD> operator%(const T val)
    {
        Matrix<T, MOD> c(a);
        for (int i = 0; i < R; i++)
            for (int j = 0; j < C; j++)
                c.a[i][j] %= val;
        return c;
    }
    T sum()
    {
        T ans = 0;
        for (auto line : a)
            for (auto x : line)
                ans = (ans + x) % MOD;
        return ans;
    }
    void print()
    {
        for (int i = 0; i < R; i++)
            for (int j = 0; j < C; j++)
                std::cout << a[i][j] << " \textbackslash{}n"[j == C - 1];
    }
};
// 模板结束

using namespace std;
const int N = 2e5 + 10;
const int inf = 1 << 30;
const long long llinf = 1ll << 60;
const double PI = acos(-1);

#define lowbit(x) (x & -x)
typedef long long ll;
typedef double db;
typedef pair<int, int> pii;
typedef pair<ll, ll> pll;
typedef pair<db, db> pdd;
typedef pair<ll, int> pli;

const ll mod = 1e9 + 7;
template <typename T>
T quickPow(T base, ll times, T initVal = 1)
{
    T ans = initVal;
    while (times)
    {
        if (times & 1)
            ans = ans * base % mod;
        base = base * base % mod;
        times >>= 1;
    }
    return ans;
}
void work()
{
    ll n, k;
    cin >> n >> k;
    vector<vector<ll>> a(n, vector<ll>(n));
    for (int i = 0; i < n; i++)
        for (int j = 0; j < n; j++)
            cin >> a[i][j];
    Matrix<ll, mod> bs(a);
    if (k == 0)
    {
        for (int i = 1; i <= n; i++)
            for (int j = 1; j <= n; j++)
                cout << (int)(i == j) << " \textbackslash{}n"[j == n];
    }
    else
        quickPow(bs, k - 1, bs).print();
}
int main()
{
    ios::sync_with_stdio(0);
    cin.tie(0);
    cout.tie(0);
    work();
}
\end{Verbatim}

\subsection{线性基}
\lstset{basicstyle=	tfamily}
\begin{Verbatim}[fontsize=\small]
struct Basis
{
#define BITSIZ 63
    typedef unsigned long long ull;
    std::array<ull, BITSIZ> arr;
    int dim = 0;
    Basis()
    {
        fill(arr.begin(), arr.end(), 0);
        dim = 0;
    }
    void insert(ull x)
    {
        for (int i = BITSIZ - 1; i >= 0; i--)
            if ((x >> i) & 1ULL)
                if (arr[i])
                    x ^= arr[i];
                else
                {
                    arr[i] = x;
                    dim++;
                    break;
                }
    }
    void join(Basis b)
    {
        for (int i = 0; i < BITSIZ; i++)
            if (b.arr[i])
                insert(b.arr[i]);
    }
    ull calcMaxNum(ull base = 0)
    {
        for (int i = BITSIZ - 1; i >= 0; i--)
            if ((arr[i] ^ base) > base)
                base ^= arr[i];
        return base;
    }
    // 返回行最简矩阵
    Basis getSimplestForm()
    {
        Basis tmp = *this;
        for (int i = BITSIZ - 1; i >= 0; i--)
        {
            if (tmp.arr[i] == 0)
                continue;
            for (int j = i - 1; j >= 0; j--)
                if ((tmp.arr[i] >> j) & 1ULL)
                    if (tmp.arr[j])
                        tmp.arr[i] ^= tmp.arr[j];
        }
        return tmp;
    }
    ull getDimention()
    {
        return dim;
    }
    // 返回第k小的异或和,  必须是行最简矩阵
    ull calcKthNum(ull k)
    {
        if (k == 0)
            return 0;
        ull ans = 0;
        for (int i = 0; k && i < BITSIZ; i++)
            if (arr[i])
            {
                if (k & 1)
                    ans ^= arr[i];
                k >>= 1;
            }
        if (k)
            return -1;
        return ans;
    }
};
\end{Verbatim}

\subsection{组合数学}
\lstset{basicstyle=	tfamily}
\begin{Verbatim}[fontsize=\small]
#include <bits/stdc++.h>

const int32_t N = 2e5 + 10;
const int64_t mod = 1e9 + 7;
int64_t fac[N], facRev[N];
auto quickPow(int64_t base, int32_t k)
{
    int64_t ans = 1LL;
    while (k)
    {
        if (k & 1)
            ans = (ans * base) % mod;
        base = (base * base) % mod;
        k >>= 1;
    }
    return ans;
}
void init()
{
    fac[0] = facRev[0] = 1;
    for (int64_t i = 1; i < N; i++)
        fac[i] = fac[i - 1] * i % mod;
    for (int64_t i = 1; i < N; i++)
        facRev[i] = quickPow(fac[i], mod - 2);
}
int64_t C(int n, int m)
{
    return fac[n] * facRev[m] % mod * facRev[n - m] % mod;
}
\end{Verbatim}

\subsection{计算几何}
\lstset{basicstyle=	tfamily}
\begin{Verbatim}[fontsize=\small]
#include <bits/stdc++.h>
using namespace std;
typedef double design_float;
const design_float PI = acos(-1);
const design_float eps = 1e-7;
struct Vector
{
    design_float x, y;
    Vector(design_float x = 0, design_float y = 0)
        : x(x), y(y)
    {
    }
    Vector operator+(const Vector &b) const { return Vector(x + b.x, y + b.y); }
    Vector operator-(const Vector &b) const { return Vector(x - b.x, y - b.y); }
    Vector operator*(const design_float &b) const { return Vector(x * b, y * b); }
    Vector operator/(const design_float &b) const { return Vector(x / b, y / b); }
    design_float operator*(const Vector &b) const { return x * b.x + y * b.y; }
    // 重定义^为叉乘
    design_float operator^(const Vector &b) const { return x * b.y - y * b.x; }
    design_float length() const { return sqrt(x * x + y * y); }
    design_float angle() const { return atan2(y, x); }
    Vector unit() const { return *this / length(); }
    // 将向量旋转angle角度,angle为弧度制
    Vector rotate(design_float angle) const
    {
        return Vector(x * cos(angle) - y * sin(angle), x * sin(angle) + y * cos(angle));
    }
};
typedef Vector Point;
struct Line
{
    Point a, b;
    Vector vec() { return b - a; }
    Line(Point a = Point(), Point b = Point())
        : a(a), b(b)
    {
    }
    // 判断点c在直线上
    bool isPointOnLine(Point c) { return ((b - a) ^ (c - a)) == 0; }
    // 判断点c在线段上
    bool isPointOnSegment(Point c)
    {
        return ((b - a) ^ (c - a)) == 0 && (c - a) * (c - b) <= 0;
    }
    // 求点c到直线的距离
    design_float distanceToLine(Point c)
    {
        return fabs(((b - a) ^ (c - a)) / (b - a).length());
    }
    // 求点c到线段的距离
    design_float distanceToSegment(Point c)
    {
        if ((c - a) * (b - a) < 0)
            return (c - a).length();
        if ((c - b) * (a - b) < 0)
            return (c - b).length();
        return distanceToLine(c);
    }
    // 求点c到线段的最近点
    Point nearestPointToSegment(Point c)
    {
        if ((c - a) * (b - a) < 0)
            return a;
        if ((c - b) * (a - b) < 0)
            return b;
        design_float r = (c - a) * (b - a) / (b - a).length();
        return a + (b - a).unit() * r;
    }
};
namespace Geometry
{
    Point rotate(Point p, Point base, design_float angle)
    {
        return (p - base).rotate(angle) + base;
    }
    Point intersection(Line l1, Line l2)
    {
        design_float s1 = (l2.b - l2.a) ^ (l1.a - l2.a);
        design_float s2 = (l2.b - l2.a) ^ (l1.b - l2.a);
        return (l1.a * s2 - l1.b * s1) / (s2 - s1);
    }
    // 输入一个点集,返回凸包
    vector<Point> getHull(vector<Point> p)
    {
        int n = p.size();
        sort(p.begin(), p.end(), [](Point a, Point b)
             { return a.x == b.x ? a.y < b.y : a.x < b.x; });
        vector<Point> res;
        for (int i = 0; i < n; i++)
        {
            while (res.size() > 1 &&)
            {
                auto v1 = res[res.size() - 1] - res[res.size() - 2];
                auto v2 = p[i] - res[res.size() - 1];
                if ((v1 ^ v2) > eps)
                    break;
                res.pop_back();
            }
            res.push_back(p[i]);
        }
        int t = res.size();
        for (int i = n - 2; i >= 0; i--)
        {
            while (res.size() > 1 &&)
            {
                auto v1 = res[res.size() - 1] - res[res.size() - 2];
                auto v2 = p[i] - res[res.size() - 1];
                if ((v1 ^ v2) > eps)
                    break;
                res.pop_back();
            }
            res.push_back(p[i]);
        }
        res.pop_back();
        return res;
    }
    // 计算凸包的边长
    design_float getHullLength(vector<Point> hull)
    {
        design_float res = 0;
        for (int i = 0; i < hull.size(); i++)
            res += (hull[i] - hull[(i + 1) % hull.size()]).length();
        return res;
    }
    // 计算凸包的面积
    design_float getHullArea(vector<Point> hull)
    {
        design_float res = 0;
        for (int i = 0; i < hull.size(); i++)
            res += (hull[i] ^ hull[(i + 1) % hull.size()]);
        return fabs(res) / 2;
    }
    bool pointOnLeft(Point p, Line l)
    {
        return ((l.b - l.a) ^ (p - l.a)) > eps;
    }
    // 求半平面交,给定若干条有向线段,算法将其排序后,以左边为半平面,返回交的凸包。
    vector<Point> getHalfPlane(vector<Line> lines)
    {
        sort(lines.begin(), lines.end(), [](auto a, auto b)
             {
            design_float angle1 = a.vec().angle(), angle2 = b.vec().angle();
            if (fabs(angle1 - angle2) < eps)
                return pointOnLeft(a.a, b);
            else
                return angle1 < angle2; });
        vector<Line> res;
        for (auto x : lines)
        {
            if (!res.empty() && fabs(x.vec().angle() - res.back().vec().angle()) < eps)
                continue;
            res.push_back(x);
        }
        lines.swap(res);
        int len = lines.size();
        int l = 1, r = 0;
        vector<int> q(len * 2);
        vector<Point> p(len * 2);
        for (int i = 0; i < len; i++)
        {
            while (l < r && !pointOnLeft(p[r], lines[i]))
                r--;
            while (l < r && !pointOnLeft(p[l + 1], lines[i]))
                l++;
            q[++r] = i;
            if (l < r && fabs((lines[q[r]].vec() ^ lines[q[r - 1]].vec())) < eps)
                if (lines[q[r]].vec() * lines[q[r - 1]].vec() < -eps)
                    return vector<Point>();
            if (l < r)
                p[r] = intersection(lines[q[r]], lines[q[r - 1]]);
        }
        while (l < r && !pointOnLeft(p[r], lines[q[l]]))
            r--;
        if (r - l <= 1)
            return vector<Point>();
        p[l] = intersection(lines[q[r]], lines[q[l]]);
        vector<Point> ans;
        while (l <= r)
        {
            ans.push_back(p[l]);
            l++;
        }
        return ans;
    }
}
int main()
{
}
\end{Verbatim}

\chapter{数据结构}
\subsection{LCT}
\lstset{basicstyle=	tfamily}
\begin{Verbatim}[fontsize=\small]
#include <algorithm>
#include <iostream>
#include <vector>
/*
 * LCT模板
 * Info 表示值,Tag 表示标记
 */
template <typename _Info, typename _Tag>
class LCT
{
public:
    LCT() : m_size(0) {}
    LCT(const int &_n, const _Info &_v = _Info())
    {
        init(_n, _v);
    }
    LCT(const std::vector<_Info> &_init)
    {
        init(_init);
    }
    void init(const int &_n, const _Info &_v = _Info())
    {
        init(std::vector<_Info>(_n, _v));
    }
    void init(const std::vector<_Info> &_init)
    {
        resize(_init.size());
        for (size_t i = 0; i < m_size; ++i)
            val[i + 1] = _init[i];
    }
    void resize(const size_t &s)
    {
        m_size = s;
        fa.resize(m_size + 1);
        siz.resize(m_size + 1);
        ch.resize(m_size + 1);
        sta.resize(m_size + 1);
        rev.resize(m_size + 1);
        val.resize(m_size + 1);
        sum.resize(m_size + 1);
        tag.resize(m_size + 1);
        for (size_t i = 0; i <= m_size; ++i)
            ch[i].resize(2);
    }
    size_t size() const { return m_size; }
    bool empty() const { return !m_size; }
    int findroot(int x)
    {
        access(x);
        _splay(x);
        while (ch[x][0])
        {
            pushdown(x);
            x = ch[x][0];
        }
        _splay(x);
        return x;
    }
    void makeroot(const int &x)
    {
        access(x);
        _splay(x);
        flip(x);
    }
    void link(const int &x, const int &y)
    {
        makeroot(x);
        if (findroot(y) != x)
            fa[x] = y;
    }
    void cut(const int &x, const int &y)
    {
        _split(x, y);
        if (findroot(y) == x && fa[y] == x && !ch[y][0])
        {
            fa[y] = ch[x][1] = 0;
            pushup(x);
        }
    }
    void set(const int &x, const _Info &v = _Info())
    {
        _splay(x);
        val[x] = v;
        pushup(x);
    }
    void change(const int &x, const int &y, const _Tag &v)
    {
        _split(x, y);
        apply(y, v);
    }
    _Info asksum(const int &x, const int &y)
    {
        _split(x, y);
        return sum[y];
    }

private:
    std::vector<int> fa, siz;
    std::vector<std::vector<int>> ch;
    std::vector<int> sta, rev;
    std::vector<_Info> val, sum;
    std::vector<_Tag> tag;
    size_t m_size;
    bool relation(const int &x) const
    {
        return ch[fa[x]][0] == x || ch[fa[x]][1] == x;
    }
    void pushup(const int &x)
    {
        sum[x] = sum[ch[x][0]] + val[x] + sum[ch[x][1]];
        siz[x] = siz[ch[x][0]] + siz[ch[x][1]] + 1;
    }
    void flip(const int &x)
    {
        std::swap(ch[x][0], ch[x][1]);
        rev[x] ^= 1;
    }
    void apply(const int &x, const _Tag &t)
    {
        val[x].apply(t);
        sum[x].apply(t);
        tag[x].apply(t);
    }
    void pushdown(const int &x)
    {
        if (rev[x])
        {
            if (ch[x][0])
                flip(ch[x][0]);
            if (ch[x][1])
                flip(ch[x][1]);
            rev[x] = 0;
        }
        if (ch[x][0])
            apply(ch[x][0], tag[x]);
        if (ch[x][1])
            apply(ch[x][1], tag[x]);
        tag[x] = _Tag();
    }
    void _rotate(const int &x)
    {
        int y = fa[x], z = fa[y], k = (ch[y][1] == x), v = ch[x][!k];
        if (relation(y))
            ch[z][ch[z][1] == y] = x;
        ch[x][!k] = y;
        ch[y][k] = v;
        if (v)
            fa[v] = y;
        fa[y] = x;
        fa[x] = z;
        pushup(y);
        pushup(x);
    }
    void _splay(int x)
    {
        int y = x, top = 1;
        sta[top] = x;
        while (relation(y))
            sta[++top] = y = fa[y];
        while (top)
            pushdown(sta[top--]);
        while (relation(x))
        {
            y = fa[x];
            top = fa[y];
            if (relation(y))
                _rotate((ch[y][1] == x) == (ch[top][1] == y) ? y : x);
            _rotate(x);
        }
        pushup(x);
    }
    void access(int x)
    {
        for (int y = 0; x; y = x, x = fa[x])
        {
            _splay(x);
            ch[x][1] = y;
            pushup(x);
        }
    }
    void _split(const int &x, const int &y)
    {
        makeroot(x);
        access(y);
        _splay(y);
    }
};
typedef long long i64;
const i64 mod = 998244353;
struct Tag
{
    i64 add, mul;
    Tag(i64 vadd = 0, i64 vmul = 1) : add(vadd), mul(vmul) {}
    void apply(const Tag &v)
    {
        mul *= v.mul;
        mul %= mod;
        add *= v.mul;
        add += v.add;
        add %= mod;
    }
};
struct Info
{
    i64 sum, siz;
    Info(i64 v = 0, i64 z = 0) : sum(v % mod), siz(z) {}
    void apply(const Tag &v)
    {
        sum = (sum * v.mul % mod + v.add * siz % mod) % mod;
    }
    Info operator+(const Info &b) const
    {
        return Info(sum + b.sum, siz + b.siz);
    }
};
LCT<Info, Tag> tr;
\end{Verbatim}

\subsection{ST表}
\lstset{basicstyle=	tfamily}
\begin{Verbatim}[fontsize=\small]
#include <vector>
#include <cstdint>
#include <algorithm>
/*
 * RMQ算法模板
 * 默认维护区间最大值
 */
namespace RMQ
{
    typedef int32_t T;
    // 维护值的类型,默认为int32_t
    std::vector<std::vector<T>> _data;
    // 预处理出log2以提速
    std::vector<int32_t> lg2;
    // 定义维护的信息,默认为max
    T func(T x, T y)
    {
        return std::max(x, y);
    }
    void init(std::vector<T> a)
    {
        int n = a.size();
        lg2 = std::vector<int>(n + 1);
        lg2[1] = 0;
        for (int i = 2; i <= n; i++)
            lg2[i] = lg2[i >> 1] + 1;
        _data = std::vector<std::vector<int>>(n, std::vector<int>(lg2[n] + 1));
        for (int i = 0; i < n; i++)
            _data[i][0] = a[i];
        for (int bit = 1; bit <= lg2[n]; bit++)
            for (int i = 0; i + (1 << bit) - 1 < n; i++)
                _data[i][bit] = func(_data[i][bit - 1], _data[i + (1 << (bit - 1))][bit - 1]);
    }
    T query(size_t l, size_t r)
    {
        int32_t bit = lg2[r - l + 1];
        return func(_data[l][bit], _data[r - (1 << bit) + 1][bit]);
    }
}
\end{Verbatim}

\subsection{二维数组(带越界检查)}
\lstset{basicstyle=	tfamily}
\begin{Verbatim}[fontsize=\small]
/*
 * 类模板实现的二维矩阵
 * 实现功能:
 * 1. 当数组越界时返回缺省值而非直接报错
 * 2. 通过成员函数获取行列数
 */

template <typename T>
class Matrix
{
private:
public:
    Matrix(size_t rows, size_t cols, const T &initial_value = T())
        : rows_(rows), cols_(cols), data_(rows, std::vector<T>(cols, initial_value)) {}

    // 重载()运算符用于访问行列列
    T &operator()(size_t row, size_t col)
    {
        if (row < 0 || row >= rows_ || col < 0 || col >= cols_)
        {
            // std::cerr << "Error: Index out of bounds\textbackslash{}n";
            //  返回一个默认值,默认为0。
            static T default_value = 0;
            return default_value;
        }
        return data_[row][col];
    }

    // 用于获取行数和列数的函数
    size_t rows() const
    {
        return rows_;
    }

    size_t cols() const
    {
        return cols_;
    }

private:
    size_t rows_;
    size_t cols_;
    std::vector<std::vector<T>> data_;
};

\end{Verbatim}

\subsection{普通Splay}
\lstset{basicstyle=	tfamily}
\begin{Verbatim}[fontsize=\small]
template <class T>
class SplayTree
{
private:
#define LEFTSON 0
#define RIGHTSON 1
    struct Node
    {
        T val;
        int son[2];
        int fa;
        int siz;
        Node()
        {
            val = T();
            son[LEFTSON] = son[RIGHTSON] = 0;
            fa = siz = 0;
        }
        Node(T val, int cnt = 1)
        {
            this->val = val;
            this->siz = 1;
            this->son[0] = this->son[1] = 0;
        }
    };
    vector<Node> tr;
    int root;
    /* 更新旋转后节点的信息 */
    void update(int x)
    {
        tr[x].siz = tr[tr[x].son[LEFTSON]].siz + tr[tr[x].son[RIGHTSON]].siz + 1;
    }
    /*返回当前编号的节点是左节点还是右节点*/
    int identify(int x)
    {
        if (tr[tr[x].fa].son[1] == x)
            return RIGHTSON;
        else
            return LEFTSON;
    }
    /*根据情况进行左旋Zig或者右旋Zag*/
    void rotate(int x)
    {
        int y = tr[x].fa, z = tr[y].fa;
        int type = identify(x), typeY = 0;
        if (z)
            tr[z].son[identify(y)] = x;
        tr[y].son[type] = tr[x].son[!type];
        if (tr[x].son[!type])
            tr[tr[x].son[!type]].fa = y;
        tr[x].son[!type] = y;
        tr[y].fa = x;
        tr[x].fa = z;
        update(y);
        update(x);
    }
    /*伸展操作,将节点旋转至目标点*/
    void splay(int x, int target = 0)
    {
        if (x == target)
            return;
        while (tr[x].fa != target)
        {
            int y = tr[x].fa;
            if (tr[y].fa == 0) // 单Zig/Zag
                rotate(x);
            else
            {
                int typeX = identify(x), typeY = identify(y);
                if (typeX == typeY) // 同向,ZigZig/ZagZag
                {
                    rotate(y);
                    rotate(x);
                }
                else // 异向,ZigZag/ZagZig
                {
                    rotate(x);
                    rotate(x);
                }
            }
        }
        if (target == 0)
            root = x;
    }

public:
    SplayTree()
    {
        tr.clear();
        tr.push_back(Node());
        root = 0;
    }
    void insert(T x)
    {
        if (root == 0)
        {
            tr.push_back(Node(x));
            root = tr.size() - 1;
        }
        else
        {
            int p = root, pf = 0;
            bool type = 0;
            while (p)
            {
                pf = p;
                type = x > tr[p].val;
                p = tr[p].son[type];
            }
            tr.push_back(Node(x));
            p = tr.size() - 1;
            tr[pf].son[type] = p;
            tr[p].fa = pf;
            splay(p);
        }
    }
    void remove(T val)
    {
        int p = root, pf = 0;
        while (p && tr[p].val != val)
        {
            pf = p;
            p = tr[p].son[val > tr[p].val];
        }
        if (!p)
        {
            if (pf)
                splay(pf);
            return;
        }
        splay(p);
        int cur = tr[p].son[0];
        if (cur == 0)
        {
            root = tr[p].son[1];
            tr[root].fa = 0;
            tr[p] = Node();
            return;
        }
        root = cur;
        while (tr[cur].son[1])
            cur = tr[cur].son[1];
        tr[cur].son[1] = tr[p].son[1];
        tr[tr[p].son[1]].fa = cur;
        tr[root].fa = 0;
        tr[p] = Node();
        splay(cur);
    }
    T find_by_rank(int rank)
    {
        int nowRank = 0, p = root;
        while (true)
        {
            if (nowRank + tr[tr[p].son[0]].siz + 1 == rank)
            {
                splay(p);
                return tr[p].val;
            }
            else if (nowRank + tr[tr[p].son[0]].siz >= rank)
                p = tr[p].son[0];
            else
            {
                nowRank += tr[tr[p].son[0]].siz + 1;
                p = tr[p].son[1];
            }
        }
    }
    int order_by_rank(T val)
    {
        int rank = 0, p = root, pf = 0;
        while (p != 0)
        {
            pf = p;
            if (tr[p].val < val)
            {
                rank += tr[tr[p].son[0]].siz + 1;
                p = tr[p].son[1];
            }
            else
                p = tr[p].son[0];
        }
        splay(pf);
        return rank + 1;
    }
    T prev(T val)
    {
        T ans = T();
        int p = root, pf = 0;
        while (p != 0)
        {
            pf = p;
            if (tr[p].val >= val)
                p = tr[p].son[0];
            else
            {
                ans = tr[p].val;
                p = tr[p].son[1];
            }
        }
        splay(pf);
        return ans;
    }
    T next(T val)
    {
        T ans = T();
        int p = root, pf = 0;
        while (p != 0)
        {
            pf = p;
            if (tr[p].val <= val)
                p = tr[p].son[1];
            else
            {
                ans = tr[p].val;
                p = tr[p].son[0];
            }
        }
        splay(pf);
        return ans;
    }
#undef LEFTSON
#undef RIGHTSON
};
\end{Verbatim}

\subsection{李超线段树}
\lstset{basicstyle=	tfamily}
\begin{Verbatim}[fontsize=\small]
/*
 * 李超线段树模板
 * 要求参数均为整数
 */
typedef long long i64;
template <int LEN>
class Lichao_SegmentTree
{
public:
    const i64 inf = 0x3f3f3f3f3f3f3f3f;
    struct Line
    {
        i64 k, b;
        Line(i64 _k = 0, i64 _b = 0) : k(_k), b(_b) {}
        i64 at(i64 x) const { return k * x + b; }
    };
    static bool less(const Line &x, const Line &y, const i64 &p)
    {
        return x.at(p) < y.at(p);
    }
    static double intersect(const Line &x, const Line &y)
    {
        return double(y.b - x.b) / double(y.k - x.k);
    }

    void init(int x1, int x2)
    {
        if (x1 > x2)
            minx = x2, maxx = x1;
        else
            minx = x1, maxx = x2;
        build(1, minx, maxx);
    }
    void insert(const Line &g, int x1, int x2) { _insert(g, x1, x2, 1, minx, maxx); }
    i64 ask(int x) const { return _getmax(x, 1, minx, maxx); }

private:
    struct Node
    {
        bool vis, has_line;
        Line f;
        Node() {}
        Node(bool _v, bool _h) : vis(_v), has_line(_h), f() {}
    };
    Node tr[LEN << 2];
    int minx, maxx;
    void build(int p, int l, int r)
    {
        tr[p] = Node(false, false);
        if (l == r)
            return;
        int mid = (l + r) >> 1;
        build(p << 1, l, mid);
        build(p << 1 | 1, mid + 1, r);
    }
    void _update(Line g, int p, int l, int r)
    {
        tr[p].vis = true;
        if (!tr[p].has_line)
        {
            tr[p].has_line = true;
            tr[p].f = g;
            return;
        }
        Line &f = tr[p].f;
        int mid = (l + r) >> 1;
        if (less(f, g, mid))
            std::swap(f, g);
        if (l == r)
            return;
        if (less(f, g, l))
            _update(g, p << 1, l, mid);
        if (less(f, g, r))
            _update(g, p << 1 | 1, mid + 1, r);
    }
    void _insert(const Line &g, int x, int y, int p, int l, int r)
    {
        tr[p].vis = true;
        if (x <= l && r <= y)
            return _update(g, p, l, r);
        int mid = (l + r) >> 1;
        if (x <= mid)
            _insert(g, x, y, p << 1, l, mid);
        if (y > mid)
            _insert(g, x, y, p << 1 | 1, mid + 1, r);
    }
    i64 _getmax(int x, int p, int l, int r) const
    {
        if (!tr[p].vis)
            return -inf;
        i64 ans = tr[p].has_line ? tr[p].f.at(x) : -inf;
        if (l == r)
            return ans;
        int mid = (l + r) >> 1;
        if (x <= mid)
            return std::max(ans, _getmax(x, p << 1, l, mid));
        else
            return std::max(ans, _getmax(x, p << 1 | 1, mid + 1, r));
        return ans;
    }
};
\end{Verbatim}

\section{查找树}
\subsection{01Treap(Trieset)}
\lstset{basicstyle=	tfamily}
\begin{Verbatim}[fontsize=\small]

#include <bits/stdc++.h>
class Trieset
{
private:
    std::vector<std::array<int, 2>> tr;
    std::vector<int> siz;

public:
    Trieset()
    {
        tr.resize(2);
        siz.resize(2);
        tr[0] = {};
        tr[1] = {};
    }
    void clear()
    {
        (*this) = Trieset();
    }
    void insert(int x, int cnt = 1)
    {
        int now = 1;
        for (int i = 31, t; i >= 0; i--)
        {
            t = (x >> i) & 1;
            if (!tr[now][t])
            {
                tr.push_back({});
                siz.push_back(0);
                tr[now][t] = tr.size() - 1;
            }
            siz[now = tr[now][t]] += cnt;
        }
    }
    bool erase(int x, int cnt = 1)
    {
        int now = 1;
        for (int i = 31, t; i >= 0; i--)
        {
            t = (x >> i) & 1;
            if (!tr[now][t])
                return 0;
            siz[now = tr[now][t]] -= cnt;
            if (siz[now] < 0)
                siz[now] = 0;
        }
        return 1;
    }
    int find_by_rank(int rank)
    {
        int ans = 0, now = 1;
        for (int i = 31; i >= 0; i--)
        {
            if (siz[tr[now][0]] < rank)
            {
                rank -= siz[tr[now][0]];
                ans |= 1 << i;
                now = tr[now][1];
            }
            else
            {
                now = tr[now][0];
            }
        }
        return ans;
    }
    int find_rank(int val)
    {
        int rank = 0, now = 1;
        for (int i = 31; i >= 0; i--)
        {
            int t = (val >> i) & 1;
            rank += t * (siz[tr[now][0]]);
            now = tr[now][t];
        }
        return rank + 1;
    }
    int prev(int val)
    {
        return find_by_rank(find_rank(val) - 1);
    }
    int next(int val)
    {
        return find_by_rank(find_rank(val + 1));
    }
} a;
\end{Verbatim}

\subsection{FHQ Treap (Reverse)}
\lstset{basicstyle=	tfamily}
\begin{Verbatim}[fontsize=\small]
#include <bits/stdc++.h>
/*
 * FHQ Treap模板(区间反转版)
 * 功能:区间反转
 * 基准模板:洛谷P3391 【模板】文艺平衡树
 * Link: https://www.luogu.com.cn/problem/P3391
 */
template <class T>
class Tree
{

private:
    struct Node
    {
        /* data */
        size_t son[2];
        int32_t rand;
        size_t size;
        T val;
        bool tag;
    };
    int32_t seed;
    size_t root, cnt;
    std::vector<Node> tr;
    int32_t rand()
    {
        return seed = ((int64_t)(seed) * 238973) % 2147483647;
    }
    int32_t add(T val)
    {
        tr.push_back(Node{{0, 0}, rand(), 1, val, 0});
        return tr.size() - 1;
    }
    void spread(int32_t now)
    {
        if (!tr[now].tag)
            return;
        std::swap(tr[now].son[0], tr[now].son[1]);
        tr[tr[now].son[0]].tag ^= 1;
        tr[tr[now].son[1]].tag ^= 1;
        tr[now].tag = 0;
    }
    void updata(size_t now)
    {
        tr[now].size = tr[tr[now].son[0]].size + tr[tr[now].son[1]].size + 1;
    }
    void split(size_t now, size_t &a, size_t &b, size_t val)
    {
        if (!now)
        {
            a = b = 0;
            return;
        }
        spread(now);
        if (tr[tr[now].son[0]].size + 1 <= val)
        {
            a = now;
            split(tr[now].son[1], tr[now].son[1], b, val - tr[tr[now].son[0]].size - 1);
        }
        else
        {
            b = now;
            split(tr[now].son[0], a, tr[now].son[0], val);
        }
        updata(now);
    }
    int32_t merge(size_t a, size_t b)
    {
        if (!a || !b)
        {
            return a | b;
        }
        if (tr[a].rand < tr[b].rand)
        {
            spread(a);
            tr[a].son[1] = merge(tr[a].son[1], b);
            updata(a);
            return a;
        }
        else
        {
            spread(b);
            tr[b].son[0] = merge(a, tr[b].son[0]);
            updata(b);
            return b;
        }
    }
    void printF(size_t now)
    {
        spread(now);
        if (tr[now].son[0])
            printF(tr[now].son[0]);
        std::cout << tr[now].val << ' ';
        if (tr[now].son[1])
            printF(tr[now].son[1]);
    }

public:
    Tree(int32_t initSeed = 19260917)
    {
        tr.push_back(Node{{0, 0},
                          rand(),
                          0,
                          std::numeric_limits<T>::min(),
                          0});
    };
    void get(std::vector<T> a)
    {
        for (auto x : a)
            root = merge(root, add(x));
    }
    void reverse(size_t l, size_t r)
    {
        size_t left, right, pos;
        split(root, left, right, r);
        split(left, left, pos, l - 1);
        tr[pos].tag ^= 1;
        root = merge(merge(left, pos), right);
    }

    void print()
    {
        printF(root);
    }
};
Tree<int32_t> a;
int32_t main()
{
    int32_t n, m;
    std::cin >> n >> m;
    std::vector<int32_t> num(n);
    for (size_t i = 0; i < n; i++)
        num[i] = i + 1;
    a.get(num);
    for (size_t i = 1, l, r; i <= m; i++)
    {
        std::cin >> l >> r;
        a.reverse(l, r);
    }
    a.print();
}
\end{Verbatim}

\subsection{树链剖分}
\lstset{basicstyle=	tfamily}
\begin{Verbatim}[fontsize=\small]
template <class T>
struct SegmentTree
{
    struct Node
    {
        T val, lazy;
        int l, r;
    };
    vector<Node> tr;
    void update(int p)
    {
        tr[p].val = (tr[p << 1].val + tr[p << 1 | 1].val);
    }
    void build(int l, int r, vector<T> &a, int p = 1)
    {

        tr[p].l = l;
        tr[p].r = r;
        tr[p].lazy = 0;
        if (l == r)
        {
            tr[p].val = a[l];
            return;
        }
        int mid = (l + r) >> 1;
        build(l, mid, a, p << 1);
        build(mid + 1, r, a, p << 1 | 1);
        update(p);
    }
    SegmentTree() {}
    SegmentTree(int n)
    {
        tr.resize(n * 4);
        vector<T> empty(n);
        build(0, n - 1, empty);
    }
    SegmentTree(vector<T> &a)
    {
        int n = a.size();
        tr.resize(n * 4);
        build(0, n - 1, a);
    }
    void pushdown(int p)
    {
        if (tr[p].lazy)
        {
            T t = tr[p].lazy;
            tr[p << 1].val += t * (tr[p << 1].r - tr[p << 1].l + 1);
            tr[p << 1].lazy += t;
            tr[p << 1 | 1].val += t * (tr[p << 1 | 1].r - tr[p << 1 | 1].l + 1);
            tr[p << 1 | 1].lazy += t;
            tr[p].lazy = 0;
        }
    }
    T query(int l, int r, int p = 1)
    {
        if (l <= tr[p].l && tr[p].r <= r)
            return tr[p].val;
        int mid = (tr[p].l + tr[p].r) >> 1;
        T sum = 0;
        pushdown(p);
        if (l <= mid)
            sum += query(l, r, p << 1);
        if (mid < r)
            sum += query(l, r, p << 1 | 1);
        return sum;
    }
    void add(int l, int r, int val, int p = 1)
    {
        if (l <= tr[p].l && tr[p].r <= r)
        {
            tr[p].val += (tr[p].r - tr[p].l + 1) * val;
            tr[p].lazy += val;
            return;
        }
        pushdown(p);
        int mid = (tr[p].l + tr[p].r) >> 1;
        if (l <= mid)
            add(l, r, val, p << 1);
        if (mid < r)
            add(l, r, val, p << 1 | 1);
        update(p);
    }
};
template <class T>
struct ChainPartition
{
    vector<vector<int>> son;
    vector<int> dep, top, siz, hson, lnk2seg, lnk2tree, f;
    vector<T> segval, treeval;
    int dfnCount;
    SegmentTree<T> tr;
    ChainPartition() {}
    ChainPartition(int n)
    {
        dfnCount = 0;
        segval.resize(n);
        f.resize(n);
        dep.resize(n);
        top.resize(n);
        siz.resize(n);
        hson.resize(n);
        lnk2seg.resize(n);
        lnk2tree.resize(n);
        son.resize(n);
        tr = SegmentTree<T>(n);
    }
    ChainPartition(vector<T> a)
    {
        int n = a.size();
        dfnCount = 0;
        segval.resize(n);
        f.resize(n);
        dep.resize(n);
        top.resize(n);
        siz.resize(n);
        hson.resize(n);
        lnk2seg.resize(n);
        lnk2tree.resize(n);
        son.resize(n);
        treeval = a;
    }
    void addEdge(int x, int y)
    {
        son[x].push_back(y);
        son[y].push_back(x);
    }
    void dfs1(int x, int fa, int depth = 1)
    {
        dep[x] = depth;
        siz[x] = 1;
        hson[x] = -1;
        f[x] = fa;
        for (int y : son[x])
        {
            if (y == fa)
                continue;
            dfs1(y, x, depth + 1);
            siz[x] += siz[y];
            if (hson[x] == -1 || siz[hson[x]] < siz[y])
                hson[x] = y;
        }
    }
    void dfs2(int x, int nowtop)
    {
        top[x] = nowtop;
        lnk2seg[x] = dfnCount++;
        lnk2tree[lnk2seg[x]] = x;
        segval[lnk2seg[x]] = treeval[x];
        if (hson[x] == -1)
            return;
        dfs2(hson[x], nowtop);
        for (int y : son[x])
        {
            if (y == f[x] || y == hson[x])
                continue;
            dfs2(y, y);
        }
    }
    void init(int root = 0)
    {
        dfs1(root, -1);
        dfs2(root, root);
        tr = SegmentTree(segval);
    }
    T querySubtree(int x)
    {
        return tr.query(lnk2seg[x], lnk2seg[x] + siz[x] - 1);
    }
    void addSubtree(int x, T val)
    {
        tr.add(lnk2seg[x], lnk2seg[x] + siz[x] - 1, val);
    }
    int getLCA(int x, int y)
    {
        while (top[x] != top[y])
        {
            if (dep[top[x]] < dep[top[y]])
                swap(x, y);
            x = f[top[x]];
        }
        if (dep[x] < dep[y])
            return x;
        else
            return y;
    }
    void add(int x, int y, T val)
    {
        while (top[x] != top[y])
        {
            if (dep[top[x]] < dep[top[y]])
                swap(x, y);
            tr.add(lnk2seg[top[x]], lnk2seg[x], val);
            x = f[top[x]];
        }
        if (dep[x] > dep[y])
            swap(x, y);
        tr.add(lnk2seg[x], lnk2seg[y], val);
    }
    T query(int x, int y)
    {
        T ans = 0;
        while (top[x] != top[y])
        {
            if (dep[top[x]] < dep[top[y]])
                swap(x, y);
            ans += tr.query(lnk2seg[top[x]], lnk2seg[x]);
            x = f[top[x]];
        }
        if (dep[x] > dep[y])
            swap(x, y);
        ans += tr.query(lnk2seg[x], lnk2seg[y]);
        return ans;
    }
};
\end{Verbatim}

\subsection{线段树}
\lstset{basicstyle=	tfamily}
\begin{Verbatim}[fontsize=\small]
#include <cstdint>
/*
 * 线段树模板
 * 默认为区间加法
 * 注意:默认下标从0开始,范围[0, n)
 */
template <class T>
struct SegmentTree
{
    struct Node
    {
        T val, lazy;
        int l, r;
    };
    vector<Node> tr;
    void updata(int p)
    {
        tr[p].val = tr[p << 1].val + tr[p << 1 | 1].val;
    }
    void build(int l, int r, vector<T> &a, int p = 1)
    {
        if (l == r)
        {
            tr[p].l = tr[p].r = l;
            tr[p].val = a[l];
            return;
        }
        int mid = (l + r) >> 1;
        build(l, mid, a, p << 1);
        build(mid + 1, r, a, p << 1 | 1);
        tr[p].l = l, tr[p].r = r;
        updata(p);
    }
    SegmentTree(int n)
    {
        tr.clear();
        tr.resize(n * 4);
        build(0, n - 1, vector<T>(n));
    }
    SegmentTree(vector<T> &a)
    {
        int n = a.size();
        tr.clear();
        tr.resize(n * 4);
        build(0, n - 1, a);
    }

    void spread(int p)
    {
        if (tr[p].lazy)
        {
            T t = tr[p].lazy;
            tr[p << 1].val += t * (tr[p << 1].r - tr[p << 1].l + 1), tr[p << 1].lazy += t;
            tr[p << 1 | 1].val += t * (tr[p << 1 | 1].r - tr[p << 1 | 1].l + 1), tr[p << 1 | 1].lazy += t;
            tr[p].lazy = 0;
        }
    }
    T query(int l, int r, int p = 1)
    {
        if (l <= tr[p].l && tr[p].r <= r)
            return tr[p].val;
        int mid = (tr[p].l + tr[p].r) >> 1;
        T sum = 0;
        spread(p);
        if (l <= mid)
            sum += query(l, r, p << 1);
        if (mid < r)
            sum += query(l, r, p << 1 | 1);
        return sum;
    }
    void add(int l, int r, int val, int p = 1)
    {
        if (l <= tr[p].l && tr[p].r <= r)
        {
            tr[p].val += (tr[p].r - tr[p].l + 1) * val;
            tr[p].lazy += val;
            return;
        }
        spread(p);
        int mid = (tr[p].l + tr[p].r) >> 1;
        if (l <= mid)
            add(l, r, val, p << 1);
        if (mid < r)
            add(l, r, val, p << 1 | 1);
        updata(p);
    }
};
\end{Verbatim}


\end{document}
